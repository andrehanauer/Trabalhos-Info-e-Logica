Aluno: André Hanauer Navarro
Data: 01/03/2019
Descrição: Descrever a utilidade dos comandos "cp", "ls" e "tree" do hydra.

Comando "cp": Tem como finalidade copiar arquivos de uma pasta para outra, utilizado de forma onde primeiro se coloca o proprio comando cp e depois de um espço adiciona-se "home/usuário/descrição da pasta onde o arquivo está/arquivo" depois de mais um espaço coloca-se "home/usuário/pasta onde deseja que o arquivo seja encaminhado".

Comando "ls": Utilizado para ver os arquivos da pasta que se está.

Comando "tree": Utilizado para uma finalidade semelhante coma do "ls" mas com este comando você consegue verificar  os arquivos que estão dentro de pastas menores dentro da sua pasta atual.
